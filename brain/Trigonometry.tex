% Created 2017-05-20 Sat 09:05
% Intended LaTeX compiler: pdflatex
\documentclass[11pt]{article}
\usepackage[utf8]{inputenc}
\usepackage[T1]{fontenc}
\usepackage{graphicx}
\usepackage{grffile}
\usepackage{longtable}
\usepackage{wrapfig}
\usepackage{rotating}
\usepackage[normalem]{ulem}
\usepackage{amsmath}
\usepackage{textcomp}
\usepackage{amssymb}
\usepackage{capt-of}
\usepackage{hyperref}
\usepackage{gensymb}
\date{\today}
\title{}
\hypersetup{
 pdfauthor={},
 pdftitle={},
 pdfkeywords={},
 pdfsubject={},
 pdfcreator={Emacs 25.2.1 (Org mode 9.0.6)}, 
 pdflang={English}}
\begin{document}

\setlength{\parindent}{0pt}

\section*{Review}
\label{sec:orgc2ee173}

Terms that we used when dealing with Trig:

\begin{itemize}
\item Angles in standard position
\item Terminal arm
\item Special triangles
\item Co-terminal angles
\item Period
\item Amplitude
\end{itemize}

We also look at transformations:

for example: \(y = a \cdot sin(k(x-d)) + c\)

Unit Circle

Radius = 1

\href{./download.jpg}{Special Triangles}

Period::One Cycle
Amplitude::Distance from axis to max

\(\theta = \frac{a}{r} = \frac{r}{r} = 1\)

How many degrees are in 1 radian?

\(57.3\degree\)

We are often expressing angles as real numbers, without units, in terms of \(\pi\)

\(\pi\) radians = \(180\degree\)

\subsection*{Convert each of the following to radians.}
\label{sec:org52e7605}

a) \(30\degree\) \((\frac{\pi}{180\degree})\)\\
= \(\frac{30\pi}{180}\)\\
= \(\frac{\pi}{6}\)

b) \(40\degree(\frac{\pi}{180\degree}\)\\
=

\subsection*{Convert each radian measure to degrees}
\label{sec:org1dd8e9b}

a) \(\frac{3\pi}{4}(\frac{180}{\pi})\)\\
= \(135\degree\)

b) 1.5 radians \((\frac{180}{\pi})\)\\
= \(85.9\degree\)

\section*{Transformations of Trigonometric Functions}
\label{sec:org9cc938b}

Remember our rules for transforming ANY function:

\((x,y) \rightarrow (\frac{x}{k} + d, ay + c)\)

\subsection*{Need to Know}
\label{sec:orgbd3cb79}

The parameters in the equations \(f(x) = a \cdot sin(k(x-d)) + c\) and \(f(x) = a \cdot cos(k(x-d)) + c\) give useful information about transformations and characteristics of the function.
\end{document}