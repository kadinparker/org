% Created 2017-04-11 Tue 13:24
% Intended LaTeX compiler: pdflatex
\documentclass[11pt]{article}
\usepackage[utf8]{inputenc}
\usepackage[T1]{fontenc}
\usepackage{graphicx}
\usepackage{grffile}
\usepackage{longtable}
\usepackage{wrapfig}
\usepackage{rotating}
\usepackage[normalem]{ulem}
\usepackage{amsmath}
\usepackage{textcomp}
\usepackage{amssymb}
\usepackage{capt-of}
\usepackage{hyperref}
\usepackage{amsmath}
\usepackage{amssymb}
\date{\today}
\title{}
\hypersetup{
 pdfauthor={},
 pdftitle={},
 pdfkeywords={},
 pdfsubject={},
 pdfcreator={Emacs 25.1.1 (Org mode 9.0.5)}, 
 pdflang={English}}
\begin{document}

\setlength{\parindent}{0pt}
\section*{Note}
\label{sec:org8b9657a}

In mathematics, you must be able to represent intervals and identify smaller sections of a relation or a set of numbers.

You have used the following inequality symbols:

> greater than\\
< less than\\
\(\geq\) greater than or equal to\\
\(\le\) less than or equal to

Linear Inequality is an inequality that contains an algebraic expression of degree 1

For example:

\(5x+3 \le 6x-2\)

We solve inequality much the same as we do equations.

\subsection*{Ex. 1 - Solve an Inequality}
\label{sec:orgc9a9002}

Solve for all values of x.

\(4x+3 \geq 15\)\\
\(4x \geq 15-3\)\\
\(4x \geq 12\)\\
\(\frac{4x}{4} \geq \frac{12}{4}\)\\
\(x \geq 3\)

We can also write out solution using set notation.

\(\{x\in \mathbb{R} \mid x \geq 3\}\)\\
\(x\in[3,\infty)\)\\

Solve for all values of x.

\(7-3x\le-2\)\\
\(-3x\le-2-7\)\\
\(-3x\le-9\)\\
\(\frac{-3x}{-3}\le\frac{-9}{-3}\)\\
\(x\geq3\)

When dividing or multiplying by a negative constant, we need to reverse the inequality sign

\(\{x \in \mathbb{R} \mid x \geq 3\}\)\\
\(x \in [3, \infty)\)

\subsection*{Ex. 3 - Solving a Double Inequality}
\label{sec:orgf16ab1a}

Solve for all values of x

\(30 \le 3(2x+4)-2(x+1) \le 46\)\\
\(30 \le 6x+12-2x-2 \le 46\)\\
\(30 \le 4x+10 \le 46\)\\
\(30-10 \le 4x+10-10 \le 46-10\)\\
\(20 \le 4x \le 36\)\\
\(\frac{20}{4} \le \frac{4x}{4} \le \frac{36}{4}\)\\
\(5 \le x \le 9\)
\end{document}